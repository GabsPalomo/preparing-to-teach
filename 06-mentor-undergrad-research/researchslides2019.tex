\documentclass{beamer}
%\documentclass{article}
%\usepackage{beamerarticle}
%\pagestyle{plain}
\input{beamdef}
\usepackage{amssymb}
\usepackage{amsmath}
\newcommand\reals{\ensuremath{\mathbb{R}}}
%\newcommand{\bfX}{{\mbox{{\bf X}}}}
\newcommand{\bfY}{{\mbox{{\bf Y}}}}
\DeclareFontShape{OT1}{cmtt}{bx}{n}{
  <5><6><7><8><9><10><10.95><12><14.4><17.28><20.74><24.88>cmttb10}{}



\listfiles

\renewcommand\emptyset{\varnothing}
%\renewcommand\cdot{\mathop{\raise2pt\hbox{$\bullet$}}}
\renewcommand\cdot{\mathop{\raise-1pt\hbox{$^{_\bullet}$}}}

\renewcommand\today{July 27, 2019}
\title[undergraduate research skills]{
Supporting Undergraduate Research Skills}
\author{Nicholas J. Horton and Mine \c Cetinkaya-Rundel}
\institute{{\large Amherst College and University of Edinburgh/RStudio}}
\date{Preparing to Teach Workshop, \today}


\mode<presentation>
{
  \usetheme{Warsaw}
  %\usetheme{Copenhagen}
  % or ...

  %\setbeamercovered{transparent}
  % or whatever (possibly just delete it)
}

\AtBeginSubsection[]
{
  %\begin{frame}<beamer>
    %\frametitle{Outline}
    %\tableofcontents[currentsection,currentsubsection]
  %\end{frame}
}



\begin{document}
\frame{\titlepage
nhorton@amherst.edu \\
https://github.com/mine-cetinkaya-rundel/preparing-to-teach}


\frame{
\frametitle{Plan}
\begin{itemize}
\item Why are research skills important?
\item Research opportunities
\item Infrastructure for research
\end{itemize}
}

\frame{
\frametitle{ASA Guidelines for Undergraduate Programs in Statistics (2014)}
\begin{itemize}
\item Graduates should be expected to write clearly, speak fluently, and construct effective visual displays and
compelling written summaries  
\item They should demonstrate ability to collaborate in teams
and to organize and manage projects
\item They should be able to communicate complex statistical methods in basic terms to managers and other audiences and visualize results in an accessible manner
\item There is pedagogical value in having students practice communication to identify gaps in their understanding.
\item Communication skills need to dovetail with students' technical and statistical knowledge: excellent communication of inappropriate or incorrect analyses is counterproductive.
\end{itemize}
}

\frame{
\frametitle{Why are research skills important?}
AAC\&U High impact practices (\url{https://www.aacu.org/leap/hips})
\begin{description}
\item[Collaborative Assignments and Projects]
Collaborative learning combines two key goals: learning to work and solve problems in the company of others, and sharpening one’s own understanding by listening seriously to the insights of others, especially those with different backgrounds and life experiences. Approaches range from study groups within a course, to team-based assignments and writing, to cooperative projects and research.
\end{description}
}

\frame{
\frametitle{Why are research skills important?}
AAC\&U High impact practices (\url{https://www.aacu.org/leap/hips})
\begin{description}
\item[Undergraduate Research]
Many colleges and universities are now providing research experiences for students in all disciplines. Undergraduate research, however, has been most prominently used in science disciplines. With strong support from the National Science Foundation and the research community, scientists are reshaping their courses to connect key concepts and questions with students’ early and active involvement in systematic investigation and research.
\end{description}
}
\frame{
\frametitle{Why are research skills important?}
AAC\&U High impact practices (\url{https://www.aacu.org/leap/hips})
\begin{description}
\item[Undergraduate Research (cont.)]
The goal is to involve students with actively contested questions, empirical observation, cutting-edge technologies, and the sense of excitement that comes from working to answer important questions.
\end{description}
}


\frame{
\frametitle{NSF Big Ideas}
\includegraphics[scale=0.48]{nsf2}
}
\frame{
\frametitle{NSF Research Experiences for Undergraduates}
\includegraphics[scale=0.48]{nsf1}
}
\frame{
\frametitle{Statistics Graduates: How do we Scale Up?}
\includegraphics[scale=0.48]{undergrad.pdf}
}

\frame{
\frametitle{Co-curricular Experiences}
\begin{itemize}
\item USPROC (Undergraduate Statistics Project Competition) \url{https://www.causeweb.org/usproc}
\begin{description}
\item[USCLAP] Undergraduate Class Project Competition
\item[USRESP] Undergraduate Research Project Competition
\end{description}
\end{itemize}
}

\frame{
\frametitle{Co-curricular Experiences (cont.)}
\includegraphics[scale=0.50]{datafest}
}

\frame{
\frametitle{Infrastructure for Research Projects (quick demos as time is available)}
\begin{itemize}
\item RPubs.com
\item github.com
\item thesisdown
\end{itemize}
}

\frame{
\frametitle{RPubs.com}
\includegraphics[scale=0.42]{rpubs1}
}

\frame{
\frametitle{RPubs.com}
\includegraphics[scale=0.44]{rpubs2}
}

\frame{
\frametitle{github.com}
\includegraphics[scale=0.36]{githubeducate}
}
\frame{
\frametitle{github.com}
\includegraphics[scale=0.44]{githubeducate2}
}
\frame{
\frametitle{thesisdown}
\includegraphics[scale=0.50]{thesisdown1}
}
\frame{
\frametitle{thesisdown}
\includegraphics[scale=0.52]{thesisdown3}
}
\frame{
\frametitle{thesisdown}
\includegraphics[scale=0.50]{thesisdown2}
}


\frame{
\frametitle{How to Sustain?}
\begin{itemize}
\item Faculty development (eCOTS, USCOTS, CAUSE, JSM, DataCamp, webinars)
\item Regional affiliations (for joint undergraduate conferences)
\item Local acknowledgemnt (value of creating these opportunities for students)
\item More support for REU's to address dramatic growth in undergraduate statistics and data science majors
\end{itemize}
}


\frame{\titlepage
nhorton@amherst.edu \\
http://nhorton.people.amherst.edu}


%\frame{
%\frametitle{Challenges and opportunities}
%\includegraphics[scale=0.420]{pval}}




\end{document}
